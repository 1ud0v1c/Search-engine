\documentclass{article}
\usepackage[utf8]{inputenc}
\usepackage[T1]{fontenc}
\usepackage[english]{babel}
\usepackage{listings}
\usepackage{color}
\usepackage{graphicx}
\usepackage{fullpage}

\usepackage{array,multirow,makecell}
\setcellgapes{1pt}
\makegapedcells
\usepackage[table]{xcolor}

\newcommand{\HRule}{\rule{\linewidth}{0.5mm}}

\lstset{
  backgroundcolor=\color{white},
  basicstyle=\footnotesize,
  breakatwhitespace=false,
  breaklines=true,
  captionpos=b,
  commentstyle=\color{red},
  deletekeywords={...},
  escapeinside={\%*}{*)},
  extendedchars=true,
  frame=false,
  keepspaces=true,
  keywordstyle=\color{blue},
  language=Java,
  literate=
  {²}{{\textsuperscript{2}}}1
  {⁴}{{\textsuperscript{4}}}1
  {⁶}{{\textsuperscript{6}}}1
  {⁸}{{\textsuperscript{8}}}1
  {€}{{\euro{}}}1
  {é}{{\'e}}1
  {è}{{\`{e}}}1
  {ê}{{\^{e}}}1
  {ë}{{\¨{e}}}1
  {É}{{\'{E}}}1
  {Ê}{{\^{E}}}1
  {û}{{\^{u}}}1
  {ù}{{\`{u}}}1
  {â}{{\^{a}}}1
  {à}{{\`{a}}}1
  {á}{{\'{a}}}1
  {ã}{{\~{a}}}1
  {Á}{{\'{A}}}1
  {Â}{{\^{A}}}1
  {Ã}{{\~{A}}}1
  {ç}{{\c{c}}}1
  {Ç}{{\c{C}}}1
  {õ}{{\~{o}}}1
  {ó}{{\'{o}}}1
  {ô}{{\^{o}}}1
  {Õ}{{\~{O}}}1
  {Ó}{{\'{O}}}1
  {Ô}{{\^{O}}}1
  {î}{{\^{i}}}1
  {Î}{{\^{I}}}1
  {í}{{\'{i}}}1
  {Í}{{\~{Í}}}1,
  morekeywords={*,...},
  numbers=left,
  numbersep=5pt,
  numberstyle=\tiny\color{black},
  rulecolor=\color{black},
  showspaces=false,
  showstringspaces=false,
  showtabs=false,
  stepnumber=1,
  stringstyle=\color{gray},
  tabsize=4,
  title=\lstname,
}

\title{ENSICAEN - 3A Info, Image\\TP 2 - Web Sémantique}
\author{\textsc{Lagarrigue} Lucie \& \textsc{Vimont} Ludovic}
\date{\today}
\makeatletter

\begin{document}

\begin{titlepage}
	\begin{center}
		\vspace*{\fill}
		\textsc{\Large \@title }
		\HRule
		\vspace{0.5cm}
		\begin{center}
			\includegraphics[width=0.7\textwidth]{../data/logo.jpg}
		\end{center}
		\vspace{0.5cm}
		\HRule \\
		\large{\@author} \\
		\@date
		\vspace*{\fill}
	\end{center}
\end{titlepage}


\section{Structure}

Nous avons donc réaliser deux classes afin de réaliser notre implémentation. La classe
Document et la classe Word. Voici la structure de la classe Document :

\begin{lstlisting}
public class Document {
  private String documentName;
  private HashMap<Word, Double>  indexes;
  private double sumPonderationSquare;

  ...
}
\end{lstlisting}

On retrouve donc trois attributs :
\begin{itemize}
  \item Le nom du document
  \item Une HashMap contenant tout les mots du document, que l'on va associer à la valeur du
  TF-IDF de ce mot.
  \item On stock également, la valeur de la somme des pondérations au carré, comme cette
  dernière ne dépend pas des éléments de la recherche, on peut la calculer en avance afin d'
  accélerer la phase de recherche.
\end{itemize}

Concernant la classe Word, voici sa structure :

\begin{lstlisting}
public class Word {
  private String name;
  private HashMap<String, Integer>  occurences = new HashMap<>();
  ...
}
\end{lstlisting}

On y voit donc deux attributs :
\begin{itemize}
  \item Le nom du mot
  \item Le nom du document dans lequel le mot est présent et le nombre d'occurences dans ce
  document donné.
\end{itemize}

Enfin, pour réaliser la phase d'indexation, nous utilisons les trois variables suivantes :

\begin{lstlisting}
LinkedList<Document> documents = new LinkedList<Document>();
HashMap<String, Integer> allWords = new HashMap<String, Integer>();
...
LinkedList<String> stopList = hfStopList.getWords();
\end{lstlisting}

\begin{itemize}
  \item La première c'est bien sur tout simplement la liste de tout les documents que l'on
  désire indexer.
  \item La seconde, c'est une liste de tout les mots présents dans chaque document. Elle nous
  permet tout simplement de compter le nombre de fois où un mot est présent dans chaque
  document.
  \item Enfin, la dernière c'est une liste de mots contenu dans une stop-liste (celle proposée
  sur la plateforme). Lors de l'ajout d'un mot dans un document, on vérifie que ce dernier n'
  est pas présent dans la stop-list.
\end{itemize}



\section{Implémentations}

Réalisé :
- indexation
- recherche
- stop list
- ligne de commande
- ?

\section{Problèmes rencontrés}


\section{Performances}


\section{Conclusion}



\end{document}

